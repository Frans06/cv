%%^^A%% um-code-sym-commands.dtx -- part of UNICODE-MATH <wspr.io/unicode-math>
%%^^A%% Definition of "math symbol alphabet" commands such as `\symbf` and co.

% \section{Mapping in maths alphabets}
% \label{sec:symcmd}
%
%    \begin{macrocode}
%<*package>
%    \end{macrocode}
%
%
% \subsection{Setting styles}
%
% Algorithm for setting alphabet fonts.
% By default, when |range| is empty, we are in \emph{implicit} mode.
% If |range| contains the name of the math alphabet, we are in \emph{explicit}
% mode and do things slightly differently.
%
% Implicit mode:
% \begin{itemize}
% \item Try and set all of the alphabet shapes.
% \item Check for the first glyph of each alphabet to detect if the font supports each
%       alphabet shape.
% \item For alphabets that do exist, overwrite whatever’s already there.
% \item For alphabets that are not supported, \emph{do nothing}.
%       (This includes leaving the old alphabet definition in place.)
% \end{itemize}
%
% Explicit mode:
% \begin{itemize}
% \item Only set the alphabets specified.
% \item Check for the first glyph of the alphabet to detect if the font contains
%       the alphabet shape in the Unicode math plane.
% \item For Unicode math alphabets, overwrite whatever’s already there.
% \item Otherwise, use the \ascii\ glyph slots instead.
% \end{itemize}
%
%
%
% \subsection{Defining the math style macros}
%
% We call the different shapes that a math alphabet can be a `math style’.
% Note that different alphabets can exist within the same math style. E.g.,
% we call `bold’ the math style |bf| and within it there are upper and lower
% case Greek and Roman alphabets and Arabic numerals.
%
% \begin{macro}{\@@_prepare_mathstyle:n}
% \darg{math style name (e.g., \texttt{it} or \texttt{bb})}
% Define the high level math alphabet macros (\cs{mathit}, etc.) in terms of
% unicode-math definitions. Use \cs{bgroup}/\cs{egroup} so s’scripts scan the
% whole thing.
%
% The flag \cs{l_@@_mathstyle_tl} is for other applications to query the
% current math style.
%    \begin{macrocode}
\cs_new:Nn \@@_prepare_mathstyle:n
  {
    \seq_gput_right:Nn \g_@@_mathstyles_seq {#1}
    \@@_init_alphabet:n {#1}
    \cs_set_protected:cpx {sym#1}
      {
        \@@_group_begin:
          \exp_not:n
            {
              \mode_if_math:F { \exp_args:Nc \non@alpherr {sym#1} }
              \tl_set:Nn \l_@@_mathstyle_tl {#1}
            }
          \@@_switch_to:n {#1}
          \@@_mathgroup_set:n {-1}
        \@@_group_end:n
      }
  }
%    \end{macrocode}
% \end{macro}
%
%
% \begin{macro}{\@@_init_alphabet:n}
% \darg{math alphabet name (e.g., \texttt{it} or \texttt{bb})}
% This macro initialises the macros used to set up a math alphabet.
% First used when the math alphabet macro is first defined, but then used
% later when redefining a particular maths alphabet.
%    \begin{macrocode}
\cs_set:Nn \@@_init_alphabet:n
  {
    \@@_log:nx {alph-initialise} {#1}
    \tl_gclear_new:c {g_@@_switchto_#1_tl}
    \cs_set_protected:cpn {@@_switchto_#1:} { \tl_use:c {g_@@_switchto_#1_tl} }
  }
\cs_new_protected:Nn \@@_switch_to:n
  {
    \tl_use:c {g_@@_switchto_#1_tl}
  }
%    \end{macrocode}
% \end{macro}
%
% \subsection{Definition of alphabets and styles}
%
%
% The linking between named ranges and symbol style commands happens here.
% It’s currently not using all of the machinery we’re in the process of setting up above.
% Baby steps.
%    \begin{macrocode}
\cs_new:Nn \@@_default_mathalph:nnn
  {
    \prop_new:c {g_@@_named_range_#1_prop}
    \seq_gput_right:Nn \g_@@_default_mathalph_seq {{#1}{#2}{#3}}
    \prop_gput:cnn { g_@@_named_range_#1_prop } { default-alpha } {#2}
  }
%    \end{macrocode}
%    \begin{macrocode}
\@@_default_mathalph:nnn {up    } {latin,Latin,greek,Greek,num,misc} {up    }
\@@_default_mathalph:nnn {it    } {latin,Latin,greek,Greek,misc}     {it    }
\@@_default_mathalph:nnn {bb    } {latin,Latin,num,misc}             {bb    }
\@@_default_mathalph:nnn {bbit  } {misc}                             {bbit  }
\@@_default_mathalph:nnn {scr   } {latin,Latin}                      {scr   }
\@@_default_mathalph:nnn {cal   } {Latin}                            {scr   }
\@@_default_mathalph:nnn {bfcal } {Latin}                            {bfscr }
\@@_default_mathalph:nnn {frak  } {latin,Latin}                      {frak  }
\@@_default_mathalph:nnn {tt    } {latin,Latin,num}                  {tt    }
\@@_default_mathalph:nnn {sfup  } {latin,Latin,num}                  {sfup  }
\@@_default_mathalph:nnn {sfit  } {latin,Latin}                      {sfit  }
\@@_default_mathalph:nnn {bfup  } {latin,Latin,greek,Greek,num,misc} {bfup  }
\@@_default_mathalph:nnn {bfit  } {latin,Latin,greek,Greek,misc}     {bfit  }
\@@_default_mathalph:nnn {bfscr } {latin,Latin}                      {bfscr }
\@@_default_mathalph:nnn {bffrak} {latin,Latin}                      {bffrak}
\@@_default_mathalph:nnn {bfsfup} {latin,Latin,greek,Greek,num,misc} {bfsfup}
\@@_default_mathalph:nnn {bfsfit} {latin,Latin,greek,Greek,misc}     {bfsfit}
%    \end{macrocode}
%
% \subsubsection{Define symbol style commands}
% Finally, all of the `symbol styles’ commands are set up, which are the commands to access each of the named alphabet styles. There is not a one-to-one mapping between symbol style commands and named style ranges!
%    \begin{macrocode}
\clist_map_inline:nn
  {
    up, it, bfup, bfit, sfup, sfit, bfsfup, bfsfit, bfsf,
    tt, bb, bbit, scr, bfscr, cal, bfcal, frak, bffrak,
    normal, literal, sf, bf,
  }
  {
    \@@_prepare_mathstyle:n {#1}
  }
%    \end{macrocode}
%
%
% \subsubsection{New names for legacy textmath alphabet selection}
% In case a package option overwrites, say, \cs{mathbf} with \cs{symbf}.
%    \begin{macrocode}
\clist_map_inline:nn
  { rm, it, bf, sf, tt }
  { \cs_set_eq:cc { mathtext #1 } { math #1 } }
%    \end{macrocode}
% Perhaps these should actually be defined using a hypothetical unicode-math interface to creating new such styles. To come.
%
%
% \subsubsection{Replacing legacy pure-maths alphabets}
% The following are alphabets which do not have a math/text ambiguity.
%    \begin{macrocode}
\clist_map_inline:nn
  {
    normal, bb , bbit, scr, bfscr, cal, bfcal, frak, bffrak, tt,
    bfup, bfit, sfup, sfit, bfsfup, bfsfit, bfsf
  }
  {
    \cs_set:cpx { math #1 } { \exp_not:c { sym #1 } }
  }
%    \end{macrocode}
%
%
% \subsubsection{New commands for ambiguous alphabets}
%
%    \begin{macrocode}
\AtBeginDocument
  {
    \clist_map_inline:nn
      { rm, it, bf, sf, tt }
      {
        \cs_set_protected:cpx { math #1 }
        {
          \exp_not:n { \bool_if:NTF  } \exp_not:c { g_@@_ math #1 _text_bool}
            { \exp_not:c { mathtext #1 } }
            { \exp_not:c { sym #1 } }
        }
      }
  }
%    \end{macrocode}
%
% \paragraph{Alias \cs{mathrm} as legacy name for \cs{mathup}}
%    \begin{macrocode}
\cs_set_protected:Npn \mathup { \mathrm }
\cs_set_protected:Npn \symrm  { \symup  }
%    \end{macrocode}
%
%
%    \begin{macrocode}
%</package>
%    \end{macrocode}

\endinput

% /©
%
% ------------------------------------------------
% The UNICODE-MATH package  <wspr.io/unicode-math>
% ------------------------------------------------
% This package is free software and may be redistributed and/or modified under
% the conditions of the LaTeX Project Public License, version 1.3c or higher
% (your choice): <http://www.latex-project.org/lppl/>.
% ------------------------------------------------
% Copyright 2006-2018  Will Robertson, LPPL "maintainer"
% Copyright 2010-2017  Philipp Stephani
% Copyright 2011-2017  Joseph Wright
% Copyright 2012-2015  Khaled Hosny
% ------------------------------------------------
%
% ©/
